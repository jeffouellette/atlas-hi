\newpage
%-------------------------------------------------------------------------------
\section{\File{atlasjournal.sty}}

Turn on including these definitions with the option \Option{journal=true} and off with the option \Option{journal=false}.

\begin{xtabular}{ll}
\verb|\AcPA| & \AcPA \\
\verb|\ARevNS| & \ARevNS \\
\verb|\CPC| & \CPC \\
\verb|\EPJ| & \EPJ \\
\verb|\EPJC| & \EPJC \\
\verb|\FortP| & \FortP \\
\verb|\IJMP| & \IJMP \\
\verb|\JETP| & \JETP \\
\verb|\JETPL| & \JETPL \\
\verb|\JaFi| & \JaFi \\
\verb|\JMP| & \JMP \\
\verb|\MPL| & \MPL \\
\verb|\NCim| & \NCim \\
\verb|\NIM| & \NIM \\
\verb|\NIMA| & \NIMA \\
\verb|\NP| & \NP \\
\verb|\NPB| & \NPB \\
\verb|\PL| & \PL \\
\verb|\PLB| & \PLB \\
\verb|\PR| & \PR \\
\verb|\PRC| & \PRC \\
\verb|\PRD| & \PRD \\
\verb|\PRL| & \PRL \\
\verb|\PRep| & \PRep \\
\verb|\RMP| & \RMP \\
\verb|\ZfP| & \ZfP \\
\verb|\collab| & \collab \\
\end{xtabular}



\newpage
%-------------------------------------------------------------------------------
\section{\File{atlasmisc.sty}}

Turn on including these definitions with the option \Option{misc=true} and off with the option \Option{misc=false}.

\begin{xtabular}{ll}
\verb|\pT| & \pT \\
\verb|\pt| & \pt \\
\verb|\ET| & \ET \\
\verb|\eT| & \eT \\
\verb|\et| & \et \\
\verb|\HT| & \HT \\
\verb|\pTsq| & \pTsq \\
\verb|\MET| & \MET \\
\verb|\met| & \met \\
\verb|\sumET| & \sumET \\
\verb|\EjetRec| & \EjetRec \\
\verb|\PjetRec| & \PjetRec \\
\verb|\EjetTru| & \EjetTru \\
\verb|\PjetTru| & \PjetTru \\
\verb|\EjetDM| & \EjetDM \\
\verb|\Rcone| & \Rcone \\
\verb|\abseta| & \abseta \\
\verb|\Ecm| & \Ecm \\
\verb|\rts| & \rts \\
\verb|\sqs| & \sqs \\
\verb|\Nevt| & \Nevt \\
\verb|\zvtx| & \zvtx \\
\verb|\dzero| & \dzero \\
\verb|\zzsth| & \zzsth \\
\verb|\RunOne| & \RunOne \\
\verb|\RunTwo| & \RunTwo \\
\verb|\RunThr| & \RunThr \\
\verb|\kt| & \kt \\
\verb|\antikt| & \antikt \\
\verb|\Antikt| & \Antikt \\
\verb|\pileup| & \pileup \\
\verb|\Pileup| & \Pileup \\
\verb|\pileup| & \pileup \\
\verb|\Pileup| & \Pileup \\
\verb|\btag| & \btag \\
\verb|\btagged| & \btagged \\
\verb|\bquark| & \bquark \\
\verb|\bquarks| & \bquarks \\
\verb|\bjet| & \bjet \\
\verb|\bjets| & \bjets \\
\verb|\mh| & \mh \\
\verb|\mW| & \mW \\
\verb|\mZ| & \mZ \\
\verb|\mH| & \mH \\
\verb|\ACERMC| & \ACERMC \\
\verb|\ALPGEN| & \ALPGEN \\
\verb|\GEANT| & \GEANT \\
\verb|\Herwigpp| & \Herwigpp \\
\verb|\HERWIGpp| & \HERWIGpp \\
\verb|\Herwig| & \Herwig \\
\verb|\HERWIG| & \HERWIG \\
\verb|\JIMMY| & \JIMMY \\
\verb|\MADSPIN| & \MADSPIN \\
\verb|\MADGRAPH| & \MADGRAPH \\
\verb|\MGMCatNLO| & \MGMCatNLO \\
\verb|\MCatNLO| & \MCatNLO \\
\verb|\AMCatNLO| & \AMCatNLO \\
\verb|\MCFM| & \MCFM \\
\verb|\METOP| & \METOP \\
\verb|\POWHEG| & \POWHEG \\
\verb|\POWHEGBOX| & \POWHEGBOX \\
\verb|\POWPYTHIA| & \POWPYTHIA \\
\verb|\PROTOS| & \PROTOS \\
\verb|\PYTHIA| & \PYTHIA \\
\verb|\SHERPA| & \SHERPA \\
\verb|\Comphep| & \Comphep \\
\verb|\Perugia| & \Perugia \\
\verb|\Prospino| & \Prospino \\
\verb|\LO| & \LO \\
\verb|\NLO| & \NLO \\
\verb|\NLL| & \NLL \\
\verb|\NNLO| & \NNLO \\
\verb|\muF| & \muF \\
\verb|\muR| & \muR \\
\verb|\ra| & \ra \\
\verb|\la| & \la \\
\verb|\rarrow| & \rarrow \\
\verb|\larrow| & \larrow \\
\verb|\lapprox| & \lapprox \\
\verb|\rapprox| & \rapprox \\
\verb|\gam| & \gam \\
\verb|\stat| & \stat \\
\verb|\syst| & \syst \\
\verb|\radlength| & \radlength \\
\verb|\StoB| & \StoB \\
\verb|\alphas| & \alphas \\
\verb|\NF| & \NF \\
\verb|\NC| & \NC \\
\verb|\CF| & \CF \\
\verb|\CA| & \CA \\
\verb|\TF| & \TF \\
\verb|\Lms| & \Lms \\
\verb|\Lmsfive| & \Lmsfive \\
\verb|\kperp| & \kperp \\
\verb|\Vcb| & \Vcb \\
\verb|\Vub| & \Vub \\
\verb|\Vtd| & \Vtd \\
\verb|\Vts| & \Vts \\
\verb|\Vtb| & \Vtb \\
\verb|\Vcs| & \Vcs \\
\verb|\Vud| & \Vud \\
\verb|\Vus| & \Vus \\
\verb|\Vcd| & \Vcd \\
\end{xtabular}


\noindent A length \Macro{figwidth} is defined that is \SI{2}{\cm} smaller than \Macro{textwidth}.

\noindent Most Monte Carlo generators also have a form with a suffix \enquote{V}
that allows you to include the version, e.g.
\verb|\PYTHIAV{8}| to produce \PYTHIAV{8} or
\verb|\PYTHIAV{8 (v8.160)}| to produce \PYTHIAV{8 (v8.160)}.

\noindent A generic macro \verb|\twomass| is defined, so that for example
\verb|\twomass{\mu}{\mu}| produces \twomass{\mu}{\mu} and \verb|\twomass{\mu}{e}| produces \twomass{\mu}{e}.

A macro \verb|\dk| is also defined which makes it easier to write down decay chains.
For example
\begin{verbatim}
\[\eqalign{a \to & b+c\\
   & \dk & e+f \\
   && \dk g+h}
\]
\end{verbatim}
produces
\[\eqalign{a \to & b+c\cr
   & \dk & e+f \cr
   && \dk g+h}
\]
Note that \Macro{eqalign} is also redefined in this package so that \Macro{dk} works.

The following macro names have been changed:\\
\verb|\ptsq| $\to$ \verb|\pTsq|.


\newpage
%-------------------------------------------------------------------------------
\section{\File{atlasxref.sty}}

Turn on including these definitions with the option \Option{xref=true} and off with the option \Option{xref=false}.

\begin{xtabular}{ll}
\end{xtabular}


\noindent The following macros with arguments are also defined:
\begin{xtabular}{ll}
\verb|\App{1}|  & \App{1}\\
\verb|\Eqn{1}|  & \Eqn{1}\\
\verb|\Fig{1}|  & \Fig{1}\\
\verb|\Ref{1}|  & \Ref{1}\\
\verb|\Sect{1}| & \Sect{1}\\
\verb|\Tab{1}|  & \Tab{1}\\
\verb|\Apps{1}{4}| & \Apps{1}{4} \\
\verb|\Eqns{1}{4}| & \Eqns{1}{4} \\
\verb|\Figs{1}{4}| & \Figs{1}{4} \\
\verb|\Refs{1}{4}| & \Refs{1}{4} \\
\verb|\Sects{1}{4}| & \Sects{1}{4} \\
\verb|\Tabs{1}{4}| & \Tabs{1}{4} \\
\verb|\Apprange{1}{4}| & \Apprange{1}{4} \\
\verb|\Eqnrange{1}{4}| & \Eqnrange{1}{4} \\
\verb|\Figrange{1}{4}| & \Figrange{1}{4} \\
\verb|\Refrange{1}{4}| & \Refrange{1}{4} \\
\verb|\Sectrange{1}{4}| & \Sectrange{1}{4} \\
\verb|\Tabrange{1}{4}| & \Tabrange{1}{4}
\end{xtabular}

The idea is that you can adapt these definitions according to your own preferences (or those of a journal).


\newpage
%-------------------------------------------------------------------------------
\section{\File{atlasbsm.sty}}

Turn on including these definitions with the option \Option{BSM} and off with the option \Option{BSM=false}.

The macro \Macro{susy} simply puts a tilde ($\tilde{\ }$) over its argument,
e.g.\ \verb|\susy{q}| produces \susy{q}.

For \susy{q}, \susy{t}, \susy{b}, \slepton, \sel, \smu and
\stau, L and R states are defined; for stop, sbottom and stau also the
light (1) and heavy (2) states.
There are four neutralinos and two charginos defined, 
the index number unfortunately needs to be written out completely. 
For the charginos the last letter(s) indicate(s) the charge: 
\enquote{p} for $+$, \enquote{m} for $-$, and \enquote{pm} for $\pm$.

\begin{xtabular}{ll}
\verb|\Azero| & \Azero \\
\verb|\hzero| & \hzero \\
\verb|\Hzero| & \Hzero \\
\verb|\Hboson| & \Hboson \\
\verb|\Hplus| & \Hplus \\
\verb|\Hminus| & \Hminus \\
\verb|\Hpm| & \Hpm \\
\verb|\Hmp| & \Hmp \\
\verb|\ggino| & \ggino \\
\verb|\chinop| & \chinop \\
\verb|\chinom| & \chinom \\
\verb|\chinopm| & \chinopm \\
\verb|\chinomp| & \chinomp \\
\verb|\chinoonep| & \chinoonep \\
\verb|\chinoonem| & \chinoonem \\
\verb|\chinoonepm| & \chinoonepm \\
\verb|\chinotwop| & \chinotwop \\
\verb|\chinotwom| & \chinotwom \\
\verb|\chinotwopm| & \chinotwopm \\
\verb|\nino| & \nino \\
\verb|\ninoone| & \ninoone \\
\verb|\ninotwo| & \ninotwo \\
\verb|\ninothree| & \ninothree \\
\verb|\ninofour| & \ninofour \\
\verb|\gravino| & \gravino \\
\verb|\Zprime| & \Zprime \\
\verb|\Zstar| & \Zstar \\
\verb|\squark| & \squark \\
\verb|\squarkL| & \squarkL \\
\verb|\squarkR| & \squarkR \\
\verb|\gluino| & \gluino \\
\verb|\stop| & \stop \\
\verb|\stopone| & \stopone \\
\verb|\stoptwo| & \stoptwo \\
\verb|\stopL| & \stopL \\
\verb|\stopR| & \stopR \\
\verb|\sbottom| & \sbottom \\
\verb|\sbottomone| & \sbottomone \\
\verb|\sbottomtwo| & \sbottomtwo \\
\verb|\sbottomL| & \sbottomL \\
\verb|\sbottomR| & \sbottomR \\
\verb|\slepton| & \slepton \\
\verb|\sleptonL| & \sleptonL \\
\verb|\sleptonR| & \sleptonR \\
\verb|\sel| & \sel \\
\verb|\selL| & \selL \\
\verb|\selR| & \selR \\
\verb|\smu| & \smu \\
\verb|\smuL| & \smuL \\
\verb|\smuR| & \smuR \\
\verb|\stau| & \stau \\
\verb|\stauL| & \stauL \\
\verb|\stauR| & \stauR \\
\verb|\stauone| & \stauone \\
\verb|\stautwo| & \stautwo \\
\verb|\snu| & \snu \\
\end{xtabular}



\newpage
%-------------------------------------------------------------------------------
\section{\File{atlasheavyion.sty}}

Turn on including these definitions with the option \Option{hion=true} and off with the option \Option{hion=false}.
The heavy ion definitions use the package \Package{mhchem} to help with the formatting of chemical elements.
This package is included by \File{atlasheavyion.sty}.

\begin{xtabular}{ll}
\verb|\NucNuc| & \NucNuc \\
\verb|\nn| & \nn \\
\verb|\pn| & \pn \\
\verb|\np| & \np \\
\verb|\PbPb| & \PbPb \\
\verb|\AuAu| & \AuAu \\
\verb|\CuCu| & \CuCu \\
\verb|\pNuc| & \pNuc \\
\verb|\pdA| & \pdA \\
\verb|\dAu| & \dAu \\
\verb|\pPb| & \pPb \\
\verb|\Npart| & \Npart \\
\verb|\avgNpart| & \avgNpart \\
\verb|\Ncoll| & \Ncoll \\
\verb|\avgNcoll| & \avgNcoll \\
\verb|\TA| & \TA \\
\verb|\avgTA| & \avgTA \\
\verb|\TPb| & \TPb \\
\verb|\avgTPb| & \avgTPb \\
\verb|\TAA| & \TAA \\
\verb|\avgTAA| & \avgTAA \\
\verb|\TAB| & \TAB \\
\verb|\avgTAB| & \avgTAB \\
\verb|\TpPb| & \TpPb \\
\verb|\avgTpPb| & \avgTpPb \\
\verb|\Gl| & \Gl \\
\verb|\GG| & \GG \\
\verb|\sqn| & \sqn \\
\verb|\lns| & \lns \\
\verb|\sumETPb| & \sumETPb \\
\verb|\sumETp| & \sumETp \\
\verb|\sumETA| & \sumETA \\
\verb|\RAA| & \RAA \\
\verb|\RCP| & \RCP \\
\verb|\RpA| & \RpA \\
\verb|\RpPb| & \RpPb \\
\verb|\dNchdeta| & \dNchdeta \\
\verb|\dNevtdET| & \dNevtdET \\
\verb|\ystar| & \ystar \\
\verb|\ycms| & \ycms \\
\verb|\ygappb| & \ygappb \\
\verb|\ygapp| & \ygapp \\
\verb|\fgap| & \fgap \\
\end{xtabular}


%The following symbols were removed or modified with respect to the original submission
%\input{atlasheavyion-mod}


\newpage
%-------------------------------------------------------------------------------
\section{\File{atlasjetetmiss.sty}}

Turn on including these definitions with the option \Option{jetetmiss=true} and off with the option \Option{jetetmiss=false}.

\begin{xtabular}{ll}
\verb|\topo| & \topo \\
\verb|\Topo| & \Topo \\
\verb|\topos| & \topos \\
\verb|\Topos| & \Topos \\
\verb|\insitu| & \insitu \\
\verb|\Insitu| & \Insitu \\
\verb|\LS| & \LS \\
\verb|\NLOjet| & \NLOjet \\
\verb|\Fastjet| & \Fastjet \\
\verb|\TwoToTwo| & \TwoToTwo \\
\verb|\largeR| & \largeR \\
\verb|\LargeR| & \LargeR \\
\verb|\akt| & \akt \\
\verb|\Akt| & \Akt \\
\verb|\AKT| & \AKT \\
\verb|\AKTFat| & \AKTFat \\
\verb|\AKTPrune| & \AKTPrune \\
\verb|\AKTFilt| & \AKTFilt \\
\verb|\KTSix| & \KTSix \\
\verb|\ca| & \ca \\
\verb|\CamKt| & \CamKt \\
\verb|\CASix| & \CASix \\
\verb|\CAFat| & \CAFat \\
\verb|\CAPrune| & \CAPrune \\
\verb|\CAFilt| & \CAFilt \\
\verb|\htt| & \htt \\
\verb|\mcut| & \mcut \\
\verb|\Nfilt| & \Nfilt \\
\verb|\Rfilt| & \Rfilt \\
\verb|\ymin| & \ymin \\
\verb|\fcut| & \fcut \\
\verb|\Rsub| & \Rsub \\
\verb|\mufrac| & \mufrac \\
\verb|\Rcut| & \Rcut \\
\verb|\zcut| & \zcut \\
\verb|\ftile| & \ftile \\
\verb|\fem| & \fem \\
\verb|\fpres| & \fpres \\
\verb|\fhec| & \fhec \\
\verb|\ffcal| & \ffcal \\
\verb|\central| & \central \\
\verb|\ecap| & \ecap \\
\verb|\forward| & \forward \\
\verb|\Npv| & \Npv \\
\verb|\Nref| & \Nref \\
\verb|\Navg| & \Navg \\
\verb|\avgmu| & \avgmu \\
\verb|\JES| & \JES \\
\verb|\JMS| & \JMS \\
\verb|\EMJES| & \EMJES \\
\verb|\GCWJES| & \GCWJES \\
\verb|\LCWJES| & \LCWJES \\
\verb|\EM| & \EM \\
\verb|\GCW| & \GCW \\
\verb|\LCW| & \LCW \\
\verb|\GSL| & \GSL \\
\verb|\GS| & \GS \\
\verb|\MTF| & \MTF \\
\verb|\MPF| & \MPF \\
\verb|\Njet| & \Njet \\
\verb|\njet| & \njet \\
\verb|\ETjet| & \ETjet \\
\verb|\etjet| & \etjet \\
\verb|\pTavg| & \pTavg \\
\verb|\ptavg| & \ptavg \\
\verb|\pTjet| & \pTjet \\
\verb|\ptjet| & \ptjet \\
\verb|\pTcorr| & \pTcorr \\
\verb|\ptcorr| & \ptcorr \\
\verb|\pTjeti| & \pTjeti \\
\verb|\ptjeti| & \ptjeti \\
\verb|\pTrecoil| & \pTrecoil \\
\verb|\ptrecoil| & \ptrecoil \\
\verb|\pTleading| & \pTleading \\
\verb|\ptleading| & \ptleading \\
\verb|\pTjetEM| & \pTjetEM \\
\verb|\ptjetEM| & \ptjetEM \\
\verb|\pThat| & \pThat \\
\verb|\pthat| & \pthat \\
\verb|\pTprobe| & \pTprobe \\
\verb|\ptprobe| & \ptprobe \\
\verb|\pTref| & \pTref \\
\verb|\ptref| & \ptref \\
\verb|\pToff| & \pToff \\
\verb|\ptoff| & \ptoff \\
\verb|\pToffjet| & \pToffjet \\
\verb|\ptoffjet| & \ptoffjet \\
\verb|\pTZ| & \pTZ \\
\verb|\ptZ| & \ptZ \\
\verb|\pTtrue| & \pTtrue \\
\verb|\pttrue| & \pttrue \\
\verb|\pTtruth| & \pTtruth \\
\verb|\pttruth| & \pttruth \\
\verb|\pTreco| & \pTreco \\
\verb|\ptreco| & \ptreco \\
\verb|\pTtrk| & \pTtrk \\
\verb|\pttrk| & \pttrk \\
\verb|\ptrk| & \ptrk \\
\verb|\pTtrkjet| & \pTtrkjet \\
\verb|\pttrkjet| & \pttrkjet \\
\verb|\ntrk| & \ntrk \\
\verb|\EoverP| & \EoverP \\
\verb|\Etrue| & \Etrue \\
\verb|\Etruth| & \Etruth \\
\verb|\Ecalo| & \Ecalo \\
\verb|\EcaloEM| & \EcaloEM \\
\verb|\asym| & \asym \\
\verb|\Response| & \Response \\
\verb|\Rcalo| & \Rcalo \\
\verb|\Rcalom| & \Rcalom \\
\verb|\RcaloEM| & \RcaloEM \\
\verb|\RMPF| & \RMPF \\
\verb|\EcaloCALIB| & \EcaloCALIB \\
\verb|\RcaloCALIB| & \RcaloCALIB \\
\verb|\EcaloEMJES| & \EcaloEMJES \\
\verb|\RcaloEMJES| & \RcaloEMJES \\
\verb|\EcaloGCWJES| & \EcaloGCWJES \\
\verb|\RcaloGCWJES| & \RcaloGCWJES \\
\verb|\EcaloLCWJES| & \EcaloLCWJES \\
\verb|\RcaloLCWJES| & \RcaloLCWJES \\
\verb|\Rtrack| & \Rtrack \\
\verb|\rtrk| & \rtrk \\
\verb|\Rtrk| & \Rtrk \\
\verb|\rtrackjet| & \rtrackjet \\
\verb|\rtrackjetiso| & \rtrackjetiso \\
\verb|\rtrackjetnoniso| & \rtrackjetnoniso \\
\verb|\rtrackjetisoratio| & \rtrackjetisoratio \\
\verb|\gammajet| & \gammajet \\
\verb|\deltaphijetgamma| & \deltaphijetgamma \\
\verb|\rapjet| & \rapjet \\
\verb|\etajet| & \etajet \\
\verb|\phijet| & \phijet \\
\verb|\etadet| & \etadet \\
\verb|\etatrk| & \etatrk \\
\verb|\Rmin| & \Rmin \\
\verb|\DeltaR| & \DeltaR \\
\verb|\DetaDphi| & \DetaDphi \\
\verb|\Deta| & \Deta \\
\verb|\Drap| & \Drap \\
\verb|\DetaOneTwo| & \DetaOneTwo \\
\verb|\DyDphi| & \DyDphi \\
\verb|\DeltaRdef| & \DeltaRdef \\
\verb|\DeltaRydef| & \DeltaRydef \\
\verb|\DeltaRtrk| & \DeltaRtrk \\
\verb|\JVF| & \JVF \\
\verb|\cJVF| & \cJVF \\
\verb|\RpT| & \RpT \\
\verb|\JVT| & \JVT \\
\verb|\ghostpt| & \ghostpt \\
\verb|\ghostptavg| & \ghostptavg \\
\verb|\ghostfm| & \ghostfm \\
\verb|\ghostfmi| & \ghostfmi \\
\verb|\ghostdensity| & \ghostdensity \\
\verb|\ghostrho| & \ghostrho \\
\verb|\Aghost| & \Aghost \\
\verb|\Amu| & \Amu \\
\verb|\Amui| & \Amui \\
\verb|\jetarea| & \jetarea \\
\verb|\jetareafm| & \jetareafm \\
\verb|\jetareai| & \jetareai \\
\verb|\Rkt| & \Rkt \\
\verb|\pTmuslope| & \pTmuslope \\
\verb|\ptmuslope| & \ptmuslope \\
\verb|\pTnpvslope| & \pTnpvslope \\
\verb|\ptnpvslope| & \ptnpvslope \\
\verb|\pTmuunc| & \pTmuunc \\
\verb|\ptmuunc| & \ptmuunc \\
\verb|\pTnpvunc| & \pTnpvunc \\
\verb|\ptnpvunc| & \ptnpvunc \\
\verb|\sumPt| & \sumPt \\
\verb|\sumpt| & \sumpt \\
\verb|\sumpTtrk| & \sumpTtrk \\
\verb|\sumpttrk| & \sumpttrk \\
\verb|\nPUtrk| & \nPUtrk \\
\verb|\mjet| & \mjet \\
\verb|\mlead| & \mlead \\
\verb|\mleadavg| & \mleadavg \\
\verb|\Mjet| & \Mjet \\
\verb|\massjet| & \massjet \\
\verb|\masscorr| & \masscorr \\
\verb|\mthresh| & \mthresh \\
\verb|\mjetavg| & \mjetavg \\
\verb|\masstrkjet| & \masstrkjet \\
\verb|\width| & \width \\
\verb|\wcalo| & \wcalo \\
\verb|\wtrk| & \wtrk \\
\verb|\shapeV| & \shapeV \\
\verb|\pTsubjet| & \pTsubjet \\
\verb|\ptsubjet| & \ptsubjet \\
\verb|\sjone| & \sjone \\
\verb|\sjtwo| & \sjtwo \\
\verb|\msubjone| & \msubjone \\
\verb|\msubjtwo| & \msubjtwo \\
\verb|\pTsubji| & \pTsubji \\
\verb|\ptsubji| & \ptsubji \\
\verb|\pTsubjone| & \pTsubjone \\
\verb|\ptsubjone| & \ptsubjone \\
\verb|\pTsubjtwo| & \pTsubjtwo \\
\verb|\ptsubjtwo| & \ptsubjtwo \\
\verb|\Rsubjets| & \Rsubjets \\
\verb|\DRsubjets| & \DRsubjets \\
\verb|\yij| & \yij \\
\verb|\dcut| & \dcut \\
\verb|\dmin| & \dmin \\
\verb|\dij| & \dij \\
\verb|\Dij| & \Dij \\
\verb|\Donetwo| & \Donetwo \\
\verb|\Dtwothr| & \Dtwothr \\
\verb|\yonetwo| & \yonetwo \\
\verb|\ytwothr| & \ytwothr \\
\verb|\yonetwoDef| & \yonetwoDef \\
\verb|\ytwothrDef| & \ytwothrDef \\
\verb|\xj| & \xj \\
\verb|\jetFunc| & \jetFunc \\
\verb|\tauone| & \tauone \\
\verb|\tautwo| & \tautwo \\
\verb|\tauthr| & \tauthr \\
\verb|\tauN| & \tauN \\
\verb|\tautwoone| & \tautwoone \\
\verb|\tauthrtwo| & \tauthrtwo \\
\verb|\dip| & \dip \\
\verb|\diponetwo| & \diponetwo \\
\verb|\diptwothr| & \diptwothr \\
\verb|\diponethr| & \diponethr \\
\verb|\mtaSup| & \mtaSup \\
\verb|\mcalo| & \mcalo \\
\verb|\mcomb| & \mcomb \\
\verb|\ECFOne| & \ECFOne \\
\verb|\ECFTwo| & \ECFTwo \\
\verb|\ECFThr| & \ECFThr \\
\verb|\ECFThrNorm| & \ECFThrNorm \\
\verb|\DTwo| & \DTwo \\
\verb|\CTwo| & \CTwo \\
\verb|\FoxWolfRatio| & \FoxWolfRatio \\
\verb|\PlanarFlow| & \PlanarFlow \\
\verb|\Angularity| & \Angularity \\
\verb|\Aplanarity| & \Aplanarity \\
\verb|\KtDR| & \KtDR \\
\verb|\Qw| & \Qw \\
\verb|\NConst| & \NConst \\
\end{xtabular}


\noindent The macro \Macro{etaRange} produces what you would expect:
\verb|\etaRange{-2.5}{+2.5}| produces \etaRange{-2.5}{+2.5} while
\verb|\AetaRange{1.0}| produces \AetaRange{1.0}.
The macro \Macro{avg} can be used for average values:
\verb|\avg{\mu}| produces \avg{\mu}.


\newpage
%-------------------------------------------------------------------------------
\section{\File{atlasmath.sty}}

Turn on including these definitions with the option \Option{math=true} and off with the option \Option{math=false}.

\begin{xtabular}{ll}
\verb|\boxsq| & \boxsq \\
\verb|\grad| & \grad \\
\end{xtabular}


\noindent The macro \Macro{spinor} is also defined.
\verb|\spinor{u}| produces \spinor{u}.


\newpage
%-------------------------------------------------------------------------------
\section{\File{atlasother.sty}}

Turn on including these definitions with the option \Option{other} and off with the option \Option{other=false}.

\begin{xtabular}{ll}
\verb|\etpt| & \etpt \\
\verb|\etptsig| & \etptsig \\
\verb|\begL| & \begL \\
\verb|\lowL| & \lowL \\
\verb|\highL| & \highL \\
\verb|\Epsb| & \Epsb \\
\verb|\Epsc| & \Epsc \\
\verb|\Mtau| & \Mtau \\
\verb|\swsq| & \swsq \\
\verb|\swel| & \swel \\
\verb|\swsqb| & \swsqb \\
\verb|\swsqon| & \swsqon \\
\verb|\gv| & \gv \\
\verb|\ga| & \ga \\
\verb|\gvbar| & \gvbar \\
\verb|\gabar| & \gabar \\
\verb|\Zzv| & \Zzv \\
\verb|\Abb| & \Abb \\
\verb|\Acc| & \Acc \\
\verb|\Aqq| & \Aqq \\
\verb|\Afb| & \Afb \\
\verb|\GZ| & \GZ \\
\verb|\GW| & \GW \\
\verb|\GH| & \GH \\
\verb|\GamHad| & \GamHad \\
\verb|\Gbb| & \Gbb \\
\verb|\Rbb| & \Rbb \\
\verb|\Gcc| & \Gcc \\
\verb|\Gvis| & \Gvis \\
\verb|\Ginv| & \Ginv \\
\end{xtabular}

